%% Generated by Sphinx.
\def\sphinxdocclass{report}
\documentclass[letterpaper,10pt,english]{sphinxmanual}
\ifdefined\pdfpxdimen
   \let\sphinxpxdimen\pdfpxdimen\else\newdimen\sphinxpxdimen
\fi \sphinxpxdimen=.75bp\relax

\PassOptionsToPackage{warn}{textcomp}
\usepackage[utf8]{inputenc}
\ifdefined\DeclareUnicodeCharacter
% support both utf8 and utf8x syntaxes
  \ifdefined\DeclareUnicodeCharacterAsOptional
    \def\sphinxDUC#1{\DeclareUnicodeCharacter{"#1}}
  \else
    \let\sphinxDUC\DeclareUnicodeCharacter
  \fi
  \sphinxDUC{00A0}{\nobreakspace}
  \sphinxDUC{2500}{\sphinxunichar{2500}}
  \sphinxDUC{2502}{\sphinxunichar{2502}}
  \sphinxDUC{2514}{\sphinxunichar{2514}}
  \sphinxDUC{251C}{\sphinxunichar{251C}}
  \sphinxDUC{2572}{\textbackslash}
\fi
\usepackage{cmap}
\usepackage[T1]{fontenc}
\usepackage{amsmath,amssymb,amstext}
\usepackage{babel}



\usepackage{times}
\expandafter\ifx\csname T@LGR\endcsname\relax
\else
% LGR was declared as font encoding
  \substitutefont{LGR}{\rmdefault}{cmr}
  \substitutefont{LGR}{\sfdefault}{cmss}
  \substitutefont{LGR}{\ttdefault}{cmtt}
\fi
\expandafter\ifx\csname T@X2\endcsname\relax
  \expandafter\ifx\csname T@T2A\endcsname\relax
  \else
  % T2A was declared as font encoding
    \substitutefont{T2A}{\rmdefault}{cmr}
    \substitutefont{T2A}{\sfdefault}{cmss}
    \substitutefont{T2A}{\ttdefault}{cmtt}
  \fi
\else
% X2 was declared as font encoding
  \substitutefont{X2}{\rmdefault}{cmr}
  \substitutefont{X2}{\sfdefault}{cmss}
  \substitutefont{X2}{\ttdefault}{cmtt}
\fi


\usepackage[Bjarne]{fncychap}
\usepackage{sphinx}

\fvset{fontsize=\small}
\usepackage{geometry}

% Include hyperref last.
\usepackage{hyperref}
% Fix anchor placement for figures with captions.
\usepackage{hypcap}% it must be loaded after hyperref.
% Set up styles of URL: it should be placed after hyperref.
\urlstyle{same}

\usepackage{sphinxmessages}
\setcounter{tocdepth}{0}



\title{TFG}
\date{May 22, 2019}
\release{1.0}
\author{Hugo Ferreira}
\newcommand{\sphinxlogo}{\vbox{}}
\renewcommand{\releasename}{Release}
\makeindex
\begin{document}

\ifdefined\shorthandoff
  \ifnum\catcode`\=\string=\active\shorthandoff{=}\fi
  \ifnum\catcode`\"=\active\shorthandoff{"}\fi
\fi

\pagestyle{empty}
\sphinxmaketitle
\pagestyle{plain}
\sphinxtableofcontents
\pagestyle{normal}
\phantomsection\label{\detokenize{index::doc}}


Contents:


\chapter{Keywords documentation!}
\label{\detokenize{keywords:module-keywords}}\label{\detokenize{keywords:keywords-documentation}}\label{\detokenize{keywords::doc}}\index{keywords (module)@\spxentry{keywords}\spxextra{module}}\phantomsection\label{\detokenize{keywords:module-keywords}}\index{keywords (module)@\spxentry{keywords}\spxextra{module}}\index{Concatenate\_candidates\_grampal() (in module keywords)@\spxentry{Concatenate\_candidates\_grampal()}\spxextra{in module keywords}}

\begin{fulllineitems}
\phantomsection\label{\detokenize{keywords:keywords.Concatenate_candidates_grampal}}\pysiglinewithargsret{\sphinxcode{\sphinxupquote{keywords.}}\sphinxbfcode{\sphinxupquote{Concatenate\_candidates\_grampal}}}{\emph{graph}, \emph{nodes}, \emph{text}}{}
Get the multiwords from the top nodes of the graph using spacy as service.
\begin{description}
\item[{Args:}] \leavevmode
graph (\sphinxtitleref{igraph}): Graph to be analyse.

nodes (\sphinxcode{\sphinxupquote{list}}): The list of top nodes.

text (\sphinxcode{\sphinxupquote{str}}): Text of origin.

\item[{Returns:}] \leavevmode
nodes (\sphinxcode{\sphinxupquote{list}}): The list of the multiwords.

\end{description}

\end{fulllineitems}

\index{Concatenate\_candidates\_spacy() (in module keywords)@\spxentry{Concatenate\_candidates\_spacy()}\spxextra{in module keywords}}

\begin{fulllineitems}
\phantomsection\label{\detokenize{keywords:keywords.Concatenate_candidates_spacy}}\pysiglinewithargsret{\sphinxcode{\sphinxupquote{keywords.}}\sphinxbfcode{\sphinxupquote{Concatenate\_candidates\_spacy}}}{\emph{graph}, \emph{nodes}, \emph{text}}{}
Get the multiwords from the top nodes of the graph using spacy as service.
\begin{description}
\item[{Args:}] \leavevmode
graph (\sphinxtitleref{igraph}): Graph to be analyse.

nodes (\sphinxcode{\sphinxupquote{list}}): The list of top nodes.

text (\sphinxcode{\sphinxupquote{str}}): Text of origin.

\item[{Returns:}] \leavevmode
nodes (\sphinxcode{\sphinxupquote{list}}): The list of the multiwords.

\end{description}

\end{fulllineitems}

\index{OrderedDict (class in keywords)@\spxentry{OrderedDict}\spxextra{class in keywords}}

\begin{fulllineitems}
\phantomsection\label{\detokenize{keywords:keywords.OrderedDict}}\pysigline{\sphinxbfcode{\sphinxupquote{class }}\sphinxcode{\sphinxupquote{keywords.}}\sphinxbfcode{\sphinxupquote{OrderedDict}}}
\sphinxcode{\sphinxupquote{OrderectDict class}}

Creates a new Dictionnary data structure that allows multiple append on the same key.
\begin{description}
\item[{Args:}] \leavevmode
Dict :The data structure Dictionnary.

\end{description}

\end{fulllineitems}

\index{Pagerank() (in module keywords)@\spxentry{Pagerank()}\spxextra{in module keywords}}

\begin{fulllineitems}
\phantomsection\label{\detokenize{keywords:keywords.Pagerank}}\pysiglinewithargsret{\sphinxcode{\sphinxupquote{keywords.}}\sphinxbfcode{\sphinxupquote{Pagerank}}}{\emph{graph}}{}
Use the Google’s pagerank algorithm  to set a value for each node.
\begin{description}
\item[{Args:}] \leavevmode
graph (\sphinxtitleref{igraph}): Graph to be analyse.

\item[{Returns:}] \leavevmode
values (\sphinxcode{\sphinxupquote{list}}): The list of values generated.

\end{description}

\end{fulllineitems}

\index{Sort\_occurences() (in module keywords)@\spxentry{Sort\_occurences()}\spxextra{in module keywords}}

\begin{fulllineitems}
\phantomsection\label{\detokenize{keywords:keywords.Sort_occurences}}\pysiglinewithargsret{\sphinxcode{\sphinxupquote{keywords.}}\sphinxbfcode{\sphinxupquote{Sort\_occurences}}}{\emph{graph}}{}
Get an array of the nodes sorted by occurence.
\begin{description}
\item[{Args:}] \leavevmode
graph (\sphinxtitleref{igraph}): Graph to be analyse.

\item[{Returns:}] \leavevmode
nodes (\sphinxcode{\sphinxupquote{list}}): The list of nodes generated.

\end{description}

\end{fulllineitems}

\index{Sort\_values() (in module keywords)@\spxentry{Sort\_values()}\spxextra{in module keywords}}

\begin{fulllineitems}
\phantomsection\label{\detokenize{keywords:keywords.Sort_values}}\pysiglinewithargsret{\sphinxcode{\sphinxupquote{keywords.}}\sphinxbfcode{\sphinxupquote{Sort\_values}}}{\emph{graph}}{}
Get an array of the nodes sorted by value.
\begin{description}
\item[{Args:}] \leavevmode
graph (\sphinxtitleref{igraph}): Graph to be analyse.

\item[{Returns:}] \leavevmode
nodes (\sphinxcode{\sphinxupquote{list}}): The list of nodes generated.

\end{description}

\end{fulllineitems}

\index{Tnodes() (in module keywords)@\spxentry{Tnodes()}\spxextra{in module keywords}}

\begin{fulllineitems}
\phantomsection\label{\detokenize{keywords:keywords.Tnodes}}\pysiglinewithargsret{\sphinxcode{\sphinxupquote{keywords.}}\sphinxbfcode{\sphinxupquote{Tnodes}}}{\emph{graph}, \emph{T}}{}
Extract the T nodes with higher values.
\begin{description}
\item[{Args:}] \leavevmode
graph (\sphinxtitleref{igraph}): Graph to be analyse.

T (\sphinxcode{\sphinxupquote{int}}): Number of nodes we want to get from the top.

\item[{Returns:}] \leavevmode
nodes (\sphinxcode{\sphinxupquote{list}}): The list of nodes generated.

\end{description}

\end{fulllineitems}

\index{create\_graph\_grampal() (in module keywords)@\spxentry{create\_graph\_grampal()}\spxextra{in module keywords}}

\begin{fulllineitems}
\phantomsection\label{\detokenize{keywords:keywords.create_graph_grampal}}\pysiglinewithargsret{\sphinxcode{\sphinxupquote{keywords.}}\sphinxbfcode{\sphinxupquote{create\_graph\_grampal}}}{\emph{text}, \emph{k=2}}{}
Create a graph with the keywords and their links using grampal as service.
\begin{description}
\item[{Args:}] \leavevmode
text (\sphinxcode{\sphinxupquote{str}}): The text of origin.

k (\sphinxcode{\sphinxupquote{int}}): The correlation value ,by default = 2.

\item[{Returns:}] \leavevmode
g (\sphinxtitleref{igraph}): The graph generated.

\end{description}

\end{fulllineitems}

\index{create\_graph\_spacy() (in module keywords)@\spxentry{create\_graph\_spacy()}\spxextra{in module keywords}}

\begin{fulllineitems}
\phantomsection\label{\detokenize{keywords:keywords.create_graph_spacy}}\pysiglinewithargsret{\sphinxcode{\sphinxupquote{keywords.}}\sphinxbfcode{\sphinxupquote{create\_graph\_spacy}}}{\emph{text}, \emph{k=2}}{}
Create a graph with the keywords and their links using spacy as service.
\begin{description}
\item[{Args:}] \leavevmode
text (\sphinxcode{\sphinxupquote{str}}): The text of origin.

k (\sphinxcode{\sphinxupquote{int}}): The correlation value ,by default = 2.

\item[{Returns:}] \leavevmode
g (\sphinxtitleref{igraph}): The graph generated.

\end{description}

\end{fulllineitems}

\index{custom\_tokenizer() (in module keywords)@\spxentry{custom\_tokenizer()}\spxextra{in module keywords}}

\begin{fulllineitems}
\phantomsection\label{\detokenize{keywords:keywords.custom_tokenizer}}\pysiglinewithargsret{\sphinxcode{\sphinxupquote{keywords.}}\sphinxbfcode{\sphinxupquote{custom\_tokenizer}}}{\emph{nlp}}{}
Redefine the custom tokenizer of spacy .
\begin{description}
\item[{Args:}] \leavevmode
nlp (\sphinxtitleref{nlp}): The tokenizer from spacy.

\item[{Returns:}] \leavevmode
nlp (\sphinxtitleref{nlp}): The new custom tokenizer.

\end{description}

\end{fulllineitems}

\index{print\_graph() (in module keywords)@\spxentry{print\_graph()}\spxextra{in module keywords}}

\begin{fulllineitems}
\phantomsection\label{\detokenize{keywords:keywords.print_graph}}\pysiglinewithargsret{\sphinxcode{\sphinxupquote{keywords.}}\sphinxbfcode{\sphinxupquote{print\_graph}}}{\emph{graph}, \emph{path}}{}
Print the graph generated, it was used for validation on small graph, currently unused .
\begin{description}
\item[{Args:}] \leavevmode
graph (\sphinxtitleref{igraph}): The graph to be printed.

path (\sphinxcode{\sphinxupquote{str}}): The path.

\end{description}

\end{fulllineitems}



\chapter{Grampal WS documentation!}
\label{\detokenize{ws:module-ws}}\label{\detokenize{ws:grampal-ws-documentation}}\label{\detokenize{ws::doc}}\index{ws (module)@\spxentry{ws}\spxextra{module}}\phantomsection\label{\detokenize{ws:module-ws}}\index{ws (module)@\spxentry{ws}\spxextra{module}}\index{Grampal (class in ws)@\spxentry{Grampal}\spxextra{class in ws}}

\begin{fulllineitems}
\phantomsection\label{\detokenize{ws:ws.Grampal}}\pysiglinewithargsret{\sphinxbfcode{\sphinxupquote{class }}\sphinxcode{\sphinxupquote{ws.}}\sphinxbfcode{\sphinxupquote{Grampal}}}{\emph{service=None}}{}
\sphinxcode{\sphinxupquote{Grampal service class}}

This class implements all the functionality of the Grampal ws, allowing the tokenize and analyse of a phrase
\index{analiza() (ws.Grampal method)@\spxentry{analiza()}\spxextra{ws.Grampal method}}

\begin{fulllineitems}
\phantomsection\label{\detokenize{ws:ws.Grampal.analiza}}\pysiglinewithargsret{\sphinxbfcode{\sphinxupquote{analiza}}}{\emph{phrase}}{}
Analyse a phrase using Grampal’s service
\begin{description}
\item[{Args:}] \leavevmode
phrase (\sphinxcode{\sphinxupquote{str}}): The phrase to be analyse.

\item[{Returns:}] \leavevmode
Object: The request object if successful, \sphinxtitleref{None} otherwise.
\begin{description}
\item[{The status\_code of the response can be checked:}] \leavevmode\begin{description}
\item[{\{}] \leavevmode
‘200’: ‘success’,                                       ‘404’: ‘not found’

\end{description}

\}

\end{description}

\end{description}

\end{fulllineitems}

\index{analiza\_get() (ws.Grampal method)@\spxentry{analiza\_get()}\spxextra{ws.Grampal method}}

\begin{fulllineitems}
\phantomsection\label{\detokenize{ws:ws.Grampal.analiza_get}}\pysiglinewithargsret{\sphinxbfcode{\sphinxupquote{analiza\_get}}}{\emph{phrase}}{}
GET function of the Grampal service
\begin{description}
\item[{Args:}] \leavevmode
phrase (\sphinxcode{\sphinxupquote{str}}): The phrase to be analyse.

\item[{Returns:}] \leavevmode
Object: The request object if successful, \sphinxtitleref{None} otherwise.
\begin{description}
\item[{The status\_code of the response can be checked:}] \leavevmode\begin{description}
\item[{\{}] \leavevmode
‘200’: ‘success’,                                       ‘404’: ‘not found’

\end{description}

\}

\end{description}

\end{description}

\end{fulllineitems}

\index{analiza\_post() (ws.Grampal method)@\spxentry{analiza\_post()}\spxextra{ws.Grampal method}}

\begin{fulllineitems}
\phantomsection\label{\detokenize{ws:ws.Grampal.analiza_post}}\pysiglinewithargsret{\sphinxbfcode{\sphinxupquote{analiza\_post}}}{\emph{phrase}}{}
POST function of the Grampal service
\begin{description}
\item[{Args:}] \leavevmode
phrase (\sphinxcode{\sphinxupquote{str}}): The phrase to be analyse.

\item[{Returns:}] \leavevmode
Object: The request object if successful, \sphinxtitleref{None} otherwise.
\begin{description}
\item[{The status\_code of the response can be checked:}] \leavevmode\begin{description}
\item[{\{}] \leavevmode
‘200’: ‘success’,                                       ‘404: ‘not found’                               \}

\end{description}

\end{description}

\end{description}

\end{fulllineitems}

\index{info\_lemma() (ws.Grampal method)@\spxentry{info\_lemma()}\spxextra{ws.Grampal method}}

\begin{fulllineitems}
\phantomsection\label{\detokenize{ws:ws.Grampal.info_lemma}}\pysiglinewithargsret{\sphinxbfcode{\sphinxupquote{info\_lemma}}}{\emph{phrase}}{}
Parse the response from the Grampal ws extracting the lemma information
\begin{description}
\item[{Args:}] \leavevmode
phrase: Phrase to be analyse

\item[{Returns:}] \leavevmode
String: The lemma information if successful, \sphinxtitleref{None} otherwise.

\end{description}

\end{fulllineitems}

\index{info\_orig() (ws.Grampal method)@\spxentry{info\_orig()}\spxextra{ws.Grampal method}}

\begin{fulllineitems}
\phantomsection\label{\detokenize{ws:ws.Grampal.info_orig}}\pysiglinewithargsret{\sphinxbfcode{\sphinxupquote{info\_orig}}}{\emph{phrase}}{}
Parse the response from the Grampal ws extracting the word of origin
\begin{description}
\item[{Args:}] \leavevmode
phrase: Phrase to be analyse

\item[{Returns:}] \leavevmode
String: The word of origin of the token

\end{description}

\end{fulllineitems}

\index{info\_syntactic() (ws.Grampal method)@\spxentry{info\_syntactic()}\spxextra{ws.Grampal method}}

\begin{fulllineitems}
\phantomsection\label{\detokenize{ws:ws.Grampal.info_syntactic}}\pysiglinewithargsret{\sphinxbfcode{\sphinxupquote{info\_syntactic}}}{\emph{phrase}}{}
Parse the response from the Grampal ws extracting the syntactic information
\begin{description}
\item[{Args:}] \leavevmode
phrase: Phrase to be analyse

\item[{Returns:}] \leavevmode
String: The syntactic information if successful, \sphinxtitleref{None} otherwise.

\end{description}

\end{fulllineitems}


\end{fulllineitems}



\chapter{Create\_json documentation!}
\label{\detokenize{create_json:module-create_json}}\label{\detokenize{create_json:create-json-documentation}}\label{\detokenize{create_json::doc}}\index{create\_json (module)@\spxentry{create\_json}\spxextra{module}}\phantomsection\label{\detokenize{create_json:module-create_json}}\index{create\_json (module)@\spxentry{create\_json}\spxextra{module}}\index{multiple\_json() (in module create\_json)@\spxentry{multiple\_json()}\spxextra{in module create\_json}}

\begin{fulllineitems}
\phantomsection\label{\detokenize{create_json:create_json.multiple_json}}\pysiglinewithargsret{\sphinxcode{\sphinxupquote{create\_json.}}\sphinxbfcode{\sphinxupquote{multiple\_json}}}{\emph{file\_name}}{}
Function that creates multiple json (one for every row) from the babelnet index format
\begin{description}
\item[{Args:}] \leavevmode
file\_name: (\sphinxcode{\sphinxupquote{str}}): The name of the index file.

\end{description}

\end{fulllineitems}

\index{single\_json() (in module create\_json)@\spxentry{single\_json()}\spxextra{in module create\_json}}

\begin{fulllineitems}
\phantomsection\label{\detokenize{create_json:create_json.single_json}}\pysiglinewithargsret{\sphinxcode{\sphinxupquote{create\_json.}}\sphinxbfcode{\sphinxupquote{single\_json}}}{\emph{file\_name}}{}
Function that creates a single json from the babelnet index format
\begin{description}
\item[{Args:}] \leavevmode
file\_name: (\sphinxcode{\sphinxupquote{str}}): The name of the index file.

\end{description}

\end{fulllineitems}



\chapter{Elastic\_bulk documentation!}
\label{\detokenize{elastic_bulk:module-elastic_bulk}}\label{\detokenize{elastic_bulk:elastic-bulk-documentation}}\label{\detokenize{elastic_bulk::doc}}\index{elastic\_bulk (module)@\spxentry{elastic\_bulk}\spxextra{module}}\phantomsection\label{\detokenize{elastic_bulk:module-elastic_bulk}}\index{elastic\_bulk (module)@\spxentry{elastic\_bulk}\spxextra{module}}\index{decode\_nginx\_log() (in module elastic\_bulk)@\spxentry{decode\_nginx\_log()}\spxextra{in module elastic\_bulk}}

\begin{fulllineitems}
\phantomsection\label{\detokenize{elastic_bulk:elastic_bulk.decode_nginx_log}}\pysiglinewithargsret{\sphinxcode{\sphinxupquote{elastic\_bulk.}}\sphinxbfcode{\sphinxupquote{decode\_nginx\_log}}}{\emph{\_nginx\_fd}}{}
Function that parse the source information from a json.
\begin{description}
\item[{Args:}] \leavevmode
\_nginx\_fd (\sphinxcode{\sphinxupquote{str}}): The name of the json file.

\item[{Returns:}] \leavevmode
Object: The json object generated

\end{description}

\end{fulllineitems}

\index{es\_add\_bulk() (in module elastic\_bulk)@\spxentry{es\_add\_bulk()}\spxextra{in module elastic\_bulk}}

\begin{fulllineitems}
\phantomsection\label{\detokenize{elastic_bulk:elastic_bulk.es_add_bulk}}\pysiglinewithargsret{\sphinxcode{\sphinxupquote{elastic\_bulk.}}\sphinxbfcode{\sphinxupquote{es\_add\_bulk}}}{\emph{nginx\_file}}{}
Function that bulk the information from a json.
\begin{description}
\item[{Args:}] \leavevmode
nginx\_file (\sphinxcode{\sphinxupquote{str}}): The name of the json file.

\end{description}

\end{fulllineitems}



\renewcommand{\indexname}{Python Module Index}
\begin{sphinxtheindex}
\let\bigletter\sphinxstyleindexlettergroup
\bigletter{c}
\item\relax\sphinxstyleindexentry{create\_json}\sphinxstyleindexextra{Unix, Windows}\sphinxstyleindexpageref{create_json:\detokenize{module-create_json}}
\indexspace
\bigletter{e}
\item\relax\sphinxstyleindexentry{elastic\_bulk}\sphinxstyleindexextra{Unix, Windows}\sphinxstyleindexpageref{elastic_bulk:\detokenize{module-elastic_bulk}}
\indexspace
\bigletter{k}
\item\relax\sphinxstyleindexentry{keywords}\sphinxstyleindexextra{Unix, Windows}\sphinxstyleindexpageref{keywords:\detokenize{module-keywords}}
\indexspace
\bigletter{w}
\item\relax\sphinxstyleindexentry{ws}\sphinxstyleindexextra{Unix, Windows}\sphinxstyleindexpageref{ws:\detokenize{module-ws}}
\end{sphinxtheindex}

\renewcommand{\indexname}{Index}
\printindex
\end{document}